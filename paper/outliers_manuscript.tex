\documentclass[english,man]{apa6}

\usepackage{amssymb,amsmath}
\usepackage{ifxetex,ifluatex}
\usepackage{fixltx2e} % provides \textsubscript
\ifnum 0\ifxetex 1\fi\ifluatex 1\fi=0 % if pdftex
  \usepackage[T1]{fontenc}
  \usepackage[utf8]{inputenc}
\else % if luatex or xelatex
  \ifxetex
    \usepackage{mathspec}
    \usepackage{xltxtra,xunicode}
  \else
    \usepackage{fontspec}
  \fi
  \defaultfontfeatures{Mapping=tex-text,Scale=MatchLowercase}
  \newcommand{\euro}{€}
\fi
% use upquote if available, for straight quotes in verbatim environments
\IfFileExists{upquote.sty}{\usepackage{upquote}}{}
% use microtype if available
\IfFileExists{microtype.sty}{\usepackage{microtype}}{}

% Table formatting
\usepackage{longtable, booktabs}
\usepackage{lscape}
% \usepackage[counterclockwise]{rotating}   % Landscape page setup for large tables
\usepackage{multirow}		% Table styling
\usepackage{tabularx}		% Control Column width
\usepackage[flushleft]{threeparttable}	% Allows for three part tables with a specified notes section
\usepackage{threeparttablex}            % Lets threeparttable work with longtable

% Create new environments so endfloat can handle them
% \newenvironment{ltable}
%   {\begin{landscape}\begin{center}\begin{threeparttable}}
%   {\end{threeparttable}\end{center}\end{landscape}}

\newenvironment{lltable}
  {\begin{landscape}\begin{center}\begin{ThreePartTable}}
  {\end{ThreePartTable}\end{center}\end{landscape}}

  \usepackage{ifthen} % Only add declarations when endfloat package is loaded
  \ifthenelse{\equal{\string man}{\string man}}{%
   \DeclareDelayedFloatFlavor{ThreePartTable}{table} % Make endfloat play with longtable
   % \DeclareDelayedFloatFlavor{ltable}{table} % Make endfloat play with lscape
   \DeclareDelayedFloatFlavor{lltable}{table} % Make endfloat play with lscape & longtable
  }{}%



% The following enables adjusting longtable caption width to table width
% Solution found at http://golatex.de/longtable-mit-caption-so-breit-wie-die-tabelle-t15767.html
\makeatletter
\newcommand\LastLTentrywidth{1em}
\newlength\longtablewidth
\setlength{\longtablewidth}{1in}
\newcommand\getlongtablewidth{%
 \begingroup
  \ifcsname LT@\roman{LT@tables}\endcsname
  \global\longtablewidth=0pt
  \renewcommand\LT@entry[2]{\global\advance\longtablewidth by ##2\relax\gdef\LastLTentrywidth{##2}}%
  \@nameuse{LT@\roman{LT@tables}}%
  \fi
\endgroup}


\ifxetex
  \usepackage[setpagesize=false, % page size defined by xetex
              unicode=false, % unicode breaks when used with xetex
              xetex]{hyperref}
\else
  \usepackage[unicode=true]{hyperref}
\fi
\hypersetup{breaklinks=true,
            pdfauthor={},
            pdftitle={Outrageous observations: The redheaded stepchild of data analysis},
            colorlinks=true,
            citecolor=blue,
            urlcolor=blue,
            linkcolor=black,
            pdfborder={0 0 0}}
\urlstyle{same}  % don't use monospace font for urls

\setlength{\parindent}{0pt}
%\setlength{\parskip}{0pt plus 0pt minus 0pt}

\setlength{\emergencystretch}{3em}  % prevent overfull lines

\ifxetex
  \usepackage{polyglossia}
  \setmainlanguage{}
\else
  \usepackage[english]{babel}
\fi

% Manuscript styling
\captionsetup{font=singlespacing,justification=justified}
\usepackage{csquotes}
\usepackage{upgreek}

 % Line numbering
  \usepackage{lineno}
  \linenumbers


\usepackage{tikz} % Variable definition to generate author note

% fix for \tightlist problem in pandoc 1.14
\providecommand{\tightlist}{%
  \setlength{\itemsep}{0pt}\setlength{\parskip}{0pt}}

% Essential manuscript parts
  \title{Outrageous observations: The redheaded stepchild of data analysis}

  \shorttitle{Title}


  \author{Erin M. Buchanan\textsuperscript{1}~\& Ernst-August Doelle\textsuperscript{1,2}}

  \def\affdep{{"", ""}}%
  \def\affcity{{"", ""}}%

  \affiliation{
    \vspace{0.5cm}
          \textsuperscript{1} Wilhelm-Wundt-University\\
          \textsuperscript{2} Konstanz Business School  }

  \authornote{
    \newcounter{author}
    Add complete departmental affiliations for each author here. Each new
    line herein must be indented, like this line.
    
    Enter author note here.

                      Correspondence concerning this article should be addressed to Erin M. Buchanan, Postal address. E-mail: \href{mailto:my@email.com}{\nolinkurl{my@email.com}}
                          }


  \abstract{Enter abstract here. Each new line herein must be indented, like this
line.}
  \keywords{keywords \\

    \indent Word count: X
  }





\usepackage{amsthm}
\newtheorem{theorem}{Theorem}
\newtheorem{lemma}{Lemma}
\theoremstyle{definition}
\newtheorem{definition}{Definition}
\newtheorem{corollary}{Corollary}
\newtheorem{proposition}{Proposition}
\theoremstyle{definition}
\newtheorem{example}{Example}
\theoremstyle{definition}
\newtheorem{exercise}{Exercise}
\theoremstyle{remark}
\newtheorem*{remark}{Remark}
\newtheorem*{solution}{Solution}
\begin{document}

\maketitle

\setcounter{secnumdepth}{0}



\emph{What are outliers.} Throughout psychology's assessments of
experimental procedures, participant observations have intrigued,
inspired, and confused experimenters world wide. Some of these
observations exist in such extremes that we begin to think of them as
outliers, influential observations, or fringliers (({\textbf{???}});
({\textbf{???}})). Since they were first commented on in 1777 by
Bernoulli, outliers have been defined as many things. ({\textbf{???}})
have synthesized their own definition that we feel is concurrent with
our work: \enquote{An outlier is an observation which being atypical
and/or erroneous deviates decidedly from the general behavior of
experimental data with respect to the criteria which is to be analyzed
on it.} (pg 217)

Outliers can be observed in many forms, which should be outlined to
ensure proper understanding and handling of such observations.
Researchers have separated outliers into categories in many different
ways over the years (({\textbf{???}}); ({\textbf{???}});
({\textbf{???}}); ({\textbf{???}}); ({\textbf{???}})). Many of the
categories used to describe outliers have noticeable overlap. For
instance, ({\textbf{???}}) stated that outliers come about by errors in
the way we gather data (inappropriate techniques, or experimenter
error), errors in preparation (improper hypothesis, planning, or
methods), or natural variability. Shortly after, ({\textbf{???}})
described outliers as consisting of people being included in an
experiment who aren't part of the population, legitimate datapoints that
are interesting because they do not fit the expected scheme, extreme
datapoints on error components, human error in observation/coding, and
errors in data preparation. Similarly, ({\textbf{???}}) state that
outliers arise, \enquote{due to mechanical faults, changes in system
behaviour, fraudulent behaviour, human error, instrument error or simply
through natural deviations in populations.} ({\textbf{???}}) have also
specified that outliers follow from data errors, intentional or
motivated mis-reporting, sampling error, standardization failure, bad
assumptions about distributions, or legitimate data points. Taken
together, we have delineated the following 4 types of outliers for the
purpose of this paper. 1. Data entry error, experimenter error in
processing, coding, or preparing data, as well as motivated
misreporting. 2. Participants who do not represent the intended
population. For example, a participant who uses a DVORAK or ergonomic
keyboard will not be appropriate in a study investigating the typing
speeds of typical QWERTY keyboard users. 3. Valid data points which
bring to light interesting phenomena the study was not aimed at
capturing. For instance, the 1854 cholera epidemic in London was found
by Dr.~John Snow to stem from a specific water well on Broad Street.
While the majority of individuals infected lived very close to the
contaminated well, Dr.~Snow also found an instance of cholera far away
from the well. When he rode out to investigate this outlying case, he
found that the woman in question had someone ride to that specific well
to draw water for her because she liked the taste (Vinten-Johansen,
Brody, Paneth, Rachman, \& Rip, 2003). 4. Natural deviations in the
population.

\emph{Effects of outliers} These outliers can have serious effects on
data, which can lead to imprecise data analysis (({\textbf{???}})),
confusing results (({\textbf{???}}); ({\textbf{???}})), and
inappropriate conclusions (({\textbf{???}}); ({\textbf{???}})). By
keeping outliers in a dataset, analyses are more likely to have
increased error variance (depending on sample size, ({\textbf{???}}))
and biased estimates (({\textbf{???}})) as well as reduced effect size
and power (({\textbf{???}}); ({\textbf{???}})) which can alter the
results of the analysis and lead to falsely supporting (Type I error) or
denying a claim (Type II error). Additionally, incorrect estimates of
effect lead to misleading meta-analyses or sample size estimates for
study planning. Beyond these effects on analyses and conclusions these
outliers can be inspirational to researchers and to their research
models as they can encourage the diagnosis, change, and evolution of a
research model (({\textbf{???}})). All together, these issues caused by
outliers can lead to furthering unwarranted avenues of research,
ignoring important information, and creating erroneous theories, all of
which serve to weaken the sciences.

\emph{Determination of outliers}

\emph{Report of outliers} For these reasons, we aim to stress the
importance of properly identifying outliers, their potential cause(s),
and ways to handle the extreme responses.

Further, as outlined by the APA (American Psychological Association,
2010), one should acknowledge these steps not only at the time of
analysis, but also at the time of publication. As ({\textbf{???}})
stated, no matter how we treat outlying observations, \enquote{It is
very important to say something about such observations in any but the
most summary report. At least how many observations were excluded from
the formal analysis, and why, should be given.} (pg. 1)

However, outlier report rates in studies have indicated that researchers
either fail to acknowledge outliers or fail to report their
acknowledgment of outliers (({\textbf{???}}); ({\textbf{???}})).
({\textbf{???}}) looked at Organizational Behavior/Human Resource
Management meta-analytic literature since 1987. As they were dealing
with meta-analyses, the outliers they were identifying consisted of
study coefficients, or correlations, not individual data points, as
outliers are usually thought of. In their study they found only six of
fifty meta-analyses used any outlier treatment for study coefficients.
Also, in most of these, highly subjective means were used to determine
which study coefficients were outliers. Similarly, ({\textbf{???}})
inspected 100 Industrial/Organizational Psychology personnel studies and
found no mention of outliers.

Although the APA and many researchers (({\textbf{???}}), …) have
instructed researchers to outline their data cleaning, there has been a
dearth in recent literature investigating if these instructions are
actually being followed. Therefore, the authors felt that the current
study examined the results in 27 journals and 672 articles across 13
psychological disciplines to assess if outliers are being reported.
Given previous report rates (({\textbf{???}}); ({\textbf{???}})) we
expect to find little to no reporting of outliers in these articles.
Further, we will examine how outlier reports relate to field, journal,
analysis, and sample size. We will also investigate the methods and
rational of outlier detection and removal.

\section{Methods}\label{methods}

We report how we determined our sample size, all data exclusions (if
any), all manipulations, and all measures in the study.

\emph{Fields} A list of psychological field areas was created to begin
the search for appropriate journals to include. The authors brainstormed
the list of topics (shown in Table XX) by first listing major research
areas in psychology (i.e.~cognitive, clinical, social). Second, a list
of common courses offered at large universities was consulted to add to
the list of fields. Lastly, the American Psychological Association's
list of divisions was examined for any potential missed fields. The
topic list was created to capture large fields of psychology with small
overlaps (i.e.~cognition and neuropsychology) while avoiding specific
subfields of topics (i.e.~cognition, perception, and memory).

\emph{Journals} Once these topic areas were decided, researchers used
various search sources (Google, EBSCO host databases) to find journals
that were dedicated to each broad topic. Journals were included if they
appeared to publish a wide range of articles within the selected fields.
A list of journals, publishers, and impact factors (as noted by the
journal website) was created for each field. Two journals from each
field were selected based on the following criteria: 1) high impact
factors, 2) impact factors over one a minimum, 3) a mix of publishers if
possible, and 4) availability due to university resources. These
journals are shown in Table XX.

\emph{Articles} Fifty articles from each journal were examined for data
analysis. Data collection of articles started at the last volume
publication from 2012 and progressed backwards until fifty articles had
been found. We excluded online first publications and started in 2012 to
ensure time for errata and retraction of articles. Articles were
including if they met the following criteria: 1) included data analyses,
2) included multiple participants or data-points, and 3) analyses were
based on human subjects or stimuli. Therefore, we excluded theory
articles, animal populations, and single subject designs. Based on
article review, three fields were excluded. Applied behavior analysis
articles predominantly included single-subject designs, evolutionary
psychology articles were primarily theory articles, and statistics
related journal articles were based on user created data with specific
set characteristics. Since none of these themes fit into our analysis of
understanding data screening with human subject samples, we excluded
those two fields from analyses.

\emph{Data Processing} Each article was then reviewed for key components
of data analyses. Each experiment in an article was treated separately.
For each experiment, the type of analysis, number of
participants/stimuli, and if they indicated outliers were coded. Types
of analyses were broadly coded into basic statistics (descriptive
statistics, z-scores, t-tests, and correlations), ANOVAs, regressions,
chi-squares, nonparametric statistics, modeling, and Bayesian/other
analyses. First, we coded if outliers were mentioned at all in article.
If so, we coded outliers into four types: 1) people, 2) data, 3) both,
and 4) none found. We found that a separate coding for data was
important for analyses with response time studies where individual
participants were not omitted but rather specific data trials were
eliminated. Then, the author decision on what to do with the outliers
was coded into whether they removed participants/stimuli or left these
outliers in the analysis, as well as if they tested the analyses with
and without these outliers for determination of their effect on the
study. If they removed outliers, a new sample size was recorded. Lastly,
we coded the reasoning for outlier detection as either participant based
(i.e.~3-month old infants were too fussy to be included), experimenter
based (i.e.~the experimental session was interrupted), statistical based
(i.e.~three SD from the mean), and no listed reason.

Table XX \emph{List of Fields, Journals, and Impact Factors 2012}

tableprint = matrix(NA, nrow = 31, ncol = 4)

tableprint{[}1, {]} = c(\enquote{Applied Behavior Analysis},
\enquote{Journal of Experimental Analysis of Behavior}, \enquote{Wiley},
1.39)

tableprint{[}2, {]} = c(\enquote{Applied Behavior Analysis},
\enquote{Journal of Applied Behavior Analysis}, \enquote{Wiley}, 1.19)

tableprint{[}3, {]} = c(\enquote{Clinical}, \enquote{Journal of
Consulting and Clinical Psychology}, \enquote{APA}, 4.85)

tableprint{[}4, {]} = c(\enquote{Clinical}, \enquote{Journal of Clinical
Psychology}, \enquote{Wiley}, 2.12)

tableprint{[}5, {]} = c(\enquote{Cognitive}, \enquote{Cognitive
Psychology}, \enquote{Elsevier}, 4.27)

tableprint{[}6, {]} = c(\enquote{Cognitive}, \enquote{Journal of
Experimental Psychology: Learning, Memory, and Cognition},
\enquote{APA}, 2.85)

tableprint{[}7, {]} = c(\enquote{Counseling}, \enquote{Journal of
Counseling}, \enquote{APA}, 3.23)

tableprint{[}8, {]} = c(\enquote{Counseling}, \enquote{The Counseling
Psychologist}, \enquote{Sage}, 1.82)

tableprint{[}9, {]} = c(\enquote{Developmental}, \enquote{Journal of
Experimental Child Psychology}, \enquote{Elsevier}, 3.23)

tableprint{[}10, {]} = c(\enquote{Developmental}, \enquote{Journal of
Youth and Adolescence}, \enquote{Springer}, 2.72)

tableprint{[}11, {]} = c(\enquote{Educational}, \enquote{Journal of
Educational Psychology}, \enquote{APA}, 3.08)

tableprint{[}12, {]} = c(\enquote{Educational}, \enquote{Contemporary
Educational Psychology}, \enquote{Elsevier}, 2.20)

tableprint{[}13, {]} = c(\enquote{Environmental}, \enquote{Journal of
Environmental Psychology}, \enquote{Elsevier}, 2.93)

tableprint{[}14, {]} = c(\enquote{Environmental}, \enquote{Environment
and Behavior}, \enquote{Sage}, 1.27)

tableprint{[}15, {]} = c(\enquote{Evolutionary}, \enquote{Evolution and
Human Behavior}, \enquote{Elsevier}, 3.11)

tableprint{[}16, {]} = c(\enquote{Evolutionary}, \enquote{Evolutionary
Psychology}, \enquote{Open Access}, 1.06)

tableprint{[}17, {]} = c(\enquote{Forensics}, \enquote{Psychology,
Public Policy, and Law}, \enquote{APA}, 1.93)

tableprint{[}18, {]} = c(\enquote{Forensics}, \enquote{Law and Human
Bevhavior}, \enquote{Spring}, 2.16)

tableprint{[}19, {]} = c(\enquote{Industrial Organization},
\enquote{Organizational Behavior and Human Decision Process},
\enquote{Elsevier}, 3.94)

tableprint{[}20, {]} = c(\enquote{Industrial Organization},
\enquote{Personnel Psychology}, \enquote{Wiley}, 2.93)

tableprint{[}21, {]} = c(\enquote{Neurological/Physiological},
\enquote{Neuropsychology}, \enquote{APA}, 3.82)

tableprint{[}22, {]} = c(\enquote{Neurological/Physiological},
\enquote{Cognitive, Affective, and Behavioral Neuroscience},
\enquote{Springer}, 3.57)

tableprint{[}23, {]} = c(\enquote{Social}, \enquote{Journal of
Personality and Social Psychology}, \enquote{APA}, 5.08)

tableprint{[}24, {]} = c(\enquote{Social}, \enquote{Journal of
Experimental Social Psychology}, \enquote{Elsevier}, 2.31)

tableprint{[}25, {]} = c(\enquote{Sports}, \enquote{Journal of Sport \&
Exercise Psychology}, \enquote{Human Kinetics}, 2.66)

tableprint{[}26, {]} = c(\enquote{Sports}, \enquote{Sociology of Sport
Journal}, \enquote{Human Kinetics}, 1.00)

tableprint{[}27, {]} = c(\enquote{Statistics}, \enquote{Special Section
of the Psychological Bulletin}, \enquote{APA}, 14.46)

tableprint{[}28, {]} = c(\enquote{Statistics}, \enquote{Structural
Equation Modeling}, \enquote{Taylor \& Francis}, 4.71)

tableprint{[}29, {]} = c(\enquote{Overview}, \enquote{Psychonomic
Bulletin \& Review}, \enquote{Springer}, 2.25)

tableprint{[}30, {]} = c(\enquote{Overview}, \enquote{Psychonomic
Science}, \enquote{Sage}, 4.43)

tableprint{[}31, {]} = c(\enquote{Overview}, \enquote{Psychological
Assessment}, \enquote{APA}, 2.99)

kable(tableprint, digits = 3, col.names = c(\enquote{Field},
\enquote{Journal}, \enquote{Publisher}, \enquote{Impact Factor}))
\emph{Note.} Impact factors as of tme of data collection (Spring 2013).

\subsection{Participants}\label{participants}

\subsection{Material}\label{material}

\subsection{Procedure}\label{procedure}

\subsection{Data analysis}\label{data-analysis}

We used R (3.4.2, R Core Team, 2017) for all our analyses.

\section{Results}\label{results}

\section{Discussion}\label{discussion}

\newpage

\section{References}\label{references}

\setlength{\parindent}{-0.5in} \setlength{\leftskip}{0.5in}

\hypertarget{refs}{}
\hypertarget{ref-R-base}{}
R Core Team. (2017). \emph{R: A language and environment for statistical
computing}. Vienna, Austria: R Foundation for Statistical Computing.
Retrieved from \url{https://www.R-project.org/}






\end{document}
